One of the primary goals of astrophysics is the observation of the Epoch of Reionization.
The Epoch of Reionization (EoR) is a period in the history of the universe when during
which the first luminous sources (stars, galaxies, etc.) formed and ionized the neutral gas.
Despite best efforts of observations of this period, the process by which the universe
transitioned from neutral to ionized is still largely a mystery. Next-generation reionization-era
experiments such as the Hydrogen of Epoch of Reionzation Array (HERA) and the Spectro-Photometer
for the History of the Universe, Epoch of Reionization, and Ices Explorer (SPHEREx)
look to independently probe the EoR by mapping the intensity of 21\,cm and Lyman-$\alpha$
fluctuations respectively. While each instrument promises sensitivity not yet seen
with this kind of measurement, the presence of bright radio foregrounds 4-5 orders of
magnitude larger than the 21\,cm fluctuations HERA seeks to measure and low-redshift H$\alpha$
interlopers. Cross-correlation of these signals looks to be a promising way to reduce
measurement systematics from each instrument and provide an independent confirmation
on the progression of reionization. In this paper, I present an estimation of the
feasibility of tracing the evolution of reionization using the 21\,cm-Ly$\alpha$
cross-power spectrum as measured by HERA and SPHEREx. Using the reionization modeling
code, 21cmFAST, I simulate the evolution of the EoR. I show that HERA and SPHEREx have
the ability to measure the cross-power spectrum at large scales given. In future work,
this technique will be used to put constraints on astrophysical parameters.


\begin{itemize}
\item EoR
\item 21cm Measurement and Difficulties
\item Ly$\alpha$ Measurement and Difficulties
\item Cross-Correlation
\item Results
\item Future Work
\end{itemize}

\newpage

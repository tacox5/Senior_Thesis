Despite the efforts of first-generation 21-cm experiments, the Epoch of Reionization
remains one of the most mysterious periods in the universe's history. Next-generation
intensity mapping experiments, such as HERA and SPHEREx, look to shed light on
this largely unexplored period. While each instrument should allow for an in-depth study
of reionization, synergies between instruments like these will offer independent
confirmation on reionization era physics and provide insight on the evolution of
the first luminous sources.

While this was a decent proof of concept, a number of potential avenues have the
opportunity to be explored in future work. This section BLAH, we provided a treatment
of 21-cm foregrounds in which, we avoided the foreground dominated region in
cylindrically averaged power spectrum by removing all modes within the foreground
wedge. In the future, it might be interesting to explore other methods of foreground
removal such as foreground subtraction. While it is likely that even optimistic
levels of foreground modeling and subtraction (99.9\% NICOLE'S PAPER), would dominate the error
budget of the cross-power spectrum, the relatively low noise contribution from SPHEREx
may allow the cross-power spectrum to be detectable even in cases of imperfect foreground
removal. In addition to an enhanced 21-cm foreground treatment, we ignored infrared foreground
removal (CITE PIXEL MASKING SPHEREx PAPER). This likely will not significantly
change the amplitude of the cross-power spectrum but could be added for completeness.

In spite of the fact that the cross-power spectrum turnover will remain undetectable
by HERA and SPHEREx measurements, given the observational parameters that were chosen, this
detection may still allow for constraints on cosmological and astrophysical parameters.
In future work, we would like to quantify the accuracy with which the 21cm-Ly$\alpha$
cross-power spectrum can allow us to estimate cosmological parameters. This analysis
has been done with the

In addition to Ly$\alpha$, 21-cm measurements made by HERA will prove to be
complementary to intensity mapping measurements of other spectral lines. In particular,
instruments measuring the CO(2-1) rotational line (COMAP) and the CII fine
structure line (TIME, CCAT-p, and CONCERTO) will be prime candidates for cross-correlation
with HERA in the near future. One possible extension of this work could be to model
intensity mapping from these lines during the EoR using 21cmFAST cubes and estimate the feasibility
of cross-correlating with these experiments. If the 21cm-CO or 21cm-CII cross-power
spectrum proves to be measurable, they should each provide additional confirmation
of an EoR detection and give insight into ionized bubble size and mean ionization fraction.

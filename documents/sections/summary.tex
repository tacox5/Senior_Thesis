Despite the efforts of first-generation 21\,cm experiments and deep galaxy surveys, the Epoch of Reionization
remains one of the most mysterious periods in the universe's history. Next-generation
intensity mapping experiments, such as HERA and SPHEREx, look to shed light on
this largely unobserved period in our universe's history. While each instrument should allow for an in-depth study
of reionization, synergies between instruments like these will offer independent
confirmation on reionization era physics and provide insight on the evolution of
the first luminous sources.

While this was a decent proof of concept, a number of potential avenues have the
opportunity to be explored in future work to improve its completeness. In Section \hyperref[sec:foregrounds]{3.3}, we provided a treatment
of 21\,cm foregrounds in which, we avoided the foreground dominated region in
cylindrically averaged power spectrum by removing all modes within the foreground
wedge. In the future, it might be interesting to explore other methods of foreground
removal such as foreground subtraction, or a foreground subtraction/avoidance hybrid approach.
It is likely that current best levels of foreground modeling and subtraction
(\cite{2019ApJ...884....1B}) would dominate the error
budget of the cross-power spectrum, but the relatively low noise contribution from SPHEREx
may allow the cross-power spectrum to be detectable even in cases of imperfect foreground
removal at large scales. In the future, I would like to incorporate this into the
anaylsis to determine its affect on a potential detection.

In addition to \lya, 21-cm measurements made by HERA will prove to be
complementary to intensity mapping measurements of other spectral lines. In particular,
instruments measuring the CO(2-1) rotational line (COMAP) and the CII fine-structure
line (TIME, CCAT-p, and CONCERTO) will be prime candidates for cross-correlation
with HERA in the near future, as the 21-CO and 21-CII cross-power spectra are expected to
have a similar anti-correlation as the 21-\lya\ cross-correlation coefficient. One possible extension of this work could be to model
intensity mapping from these lines during the EoR using \fastsim\ cubes and estimate the feasibility of HERA
of cross-correlating with these experiments given realistic observing strategies. If the 21cm-CO or 21cm-CII cross-power
spectrum proves to be measurable, they should each provide additional confirmation
of an EoR detection and give insight into ionized bubble size and mean ionization fraction.

Although the cross-power spectrum turn-over will likely remain undetectable
by HERA and SPHEREx measurements given these estimates, this
detection may still allow for constraints on cosmological and astrophysical parameters.
Potentially the most exciting aspect of this cross-correlation work will
be determining our ability to break down parameter degeneracies that
21\,cm and \lya\ measurements are expected to have. In future work, we would like to quantify
the accuracy with which the 21cm-\lya\ cross-power spectrum can allow us to estimate
cosmological parameters. This type of analysis has been done in the past in \cite{2014ApJ...782...66P}
with the 21\,cm power spectrum and should apply well to this work.

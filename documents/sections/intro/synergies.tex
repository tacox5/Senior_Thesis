One particularly interesting area of study is that of synergies between experiments
studying the reionization. As mentioned in previous sections, both HERA and SPHEREx
have the potential to
detect fluctuations during reionization. HERA will measure emission from neutral gas
between early galaxies, while SPHEREx will study the ionized bubbles that the
neutral gas surrounds. By combining measurements from HERA and SPHEREx, we have
the potential to trace the entire IGM, neutral and ionized, providing more
information on the progression of reionization than either measurement will
be able to alone. Specifically, the cross-power spectrum and cross-correlation
coefficients seem to very promising tools for understanding the reionization.
Given that 21\,cm emission traces the neutral medium and \lya\ observes the ionized
medium, cross-correlating these should produce a strong anti-correlation on large
spatial scales, where emission from these lines are completely separate, that smoothly transitions
to no correlation on small spatial scales, where either only 21\,cm or Ly$\alpha$
will be observed.

Cross-correlation of reionization-era measurements is proven to be feasible and
useful for understanding the progression of the EoR. In \cite{2017ApJ...836..176H},
it was found that negative correlation appears when cross-correlating 21\,cm and
\lya\ fluctuations that help break down degeneracies in parameter estimates that
show up in estimates with 21\,cm and \lya\ measurements alone. Additionally, \cite{2016MNRAS.459.2741S}
showed that cross-correlating 21\,cm and \lya\ emitters helps to improve constraints
on mean neutral fraction estimates. This techique has accomplished been done in practice using the
Greenbank Telescope cross-correlating with galaxy surveys at $z \approx 0.8$, but
a comparable measurement has never been made at reionization-era redshifts.

Past papers have demonstrated the feasibility of making a detection
of the reionzation-era 21\,cm-\lya\ cross-power spectrum, but with instruments that may be a decade or more away
from being constructed (\cite{2017ApJ...836..176H}, \cite{2017ApJ...848...52H}, \cite{2018MNRAS.479.2754K}).
While it is nearly certain that future experiments such as the Square Kilometer Array (SKA) and Cosmic Dawn Intensity
Mapper (CDIM) will have the capability of making a detection of the cross-power spectrum
opportunities for its detection may be closer than past estimates suggest.
With the construction of HERA's 350 element array to be completed in 2020 and
the launch SPHEREx targeted for the end of 2023, these two
instruments have the potential to be the first to cross-correlate intensity
mapping measurements made of reionization before the end of the next decade.

Both of these experiments have the potential to measure emission from $z \approx 6-8$
and are sensitive enough to detect the EoR on their own, but cross-correlation between
these instuments is complicated by the fact that they measure different angular
scales and are therefore sensitive to different spatial scales. They do overlap
in overlap in the spatial scales that they measure (as can be seen in Figure
\ref{fig:intensity_mapping}), so a detection of the cross-power spectrum is technically possible
given the amplitude of the signal and sensitivity that each instrument is able to achieve.

In the following section, I will estimate the amplitude of the cross-power spectrum and
calculate the cross-correlation coefficient by simulating 21\,cm and \lya\ fluctuation
fields. I will then test the ability of HERA and SPHEREx to make a detection of the
cross-power spectrum by estimating the noise on the cross-power spectrum using
reasonable estimates of each instrument's spectral and spatial resolving power and taking into
account sample variance, thermal noise variance, and 21\,cm foregrounds.

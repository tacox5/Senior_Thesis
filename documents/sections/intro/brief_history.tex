Immediately following the Big Bang, the Universe was primarily composed of a hot plasma of fundamental particles. In its early state, it was much too hot and dense to form the atoms that form the complex structures the astronomers observe today. Photons that were emitted during this early period scattered off free particles, leaving the Universe opaque to electromagnetic radiation. This lasted until roughly 400,000 years after the Big Bang, at which time the Universe had expanded and cooled sufficiently for electrons to bind to atomic nuclei forming the first atoms of Hydrogen and Helium. Once formed, photons were able to freely propagate through the intergalactic medium (IGM) as cosmic microwave background radiation.

The Cosmic Microwave Background (CMB) is arguably the best studied cosmological period in the
history of the universe. Space-based observatories such as WMAP and Planck have measured the
CMB with increasing accuracy giving cosmologists insight to the very first fractions
of a second after the Big Bang.

Much of the history of the Universe after $\sim$1 billion years the Big Bang is observable as well. Optical and infrared observatories such as those done with the Hubble Space Telescope (HST) allows astronomers to regularly observe redshift galaxies out to $z \sim 4$, while a
number of objects at $z > 7$ have been observed with with gravitational lensing. While some objects have been observed at high redshifts, little is known about the period of cosmological time between when the.

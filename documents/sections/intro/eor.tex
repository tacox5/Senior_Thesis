While modeling the physics of the Cosmic Dark Ages,

While this general picture of reionization is relatively well accepted, the details
of this picture are largely unconstrained and leave many questions unanswered.
The beginning of reionization, the time it took the first luminous sources to fully ionize
the IGM, and size and growth of early ionized bubbles are open questions which are
sought to be answered. If we are able to answer these questions, we also have the
ability to shed light on more fundamental areas of astrophysics, such as what the
first luminous objects were like and how they formed.

While current measurements of the EoR are slim, there are a few studies that allow us
to constrain the progress of reionization. Measurements of the polarization
of the CMB can help place some restriction on the timing of reionization. By assuming
reionization was instantaneous, the WMAP and Planck CMB probes used the optical depth of the
CMB to infer a reionization redshift of $z \approx 8.8$. In addition, high-redshift quasar studies provide insight to
when reionization likely came to a close. Using observations of the Gunn-Peterson trough, \cite{}
found that reionization concluded no later than $z \approx 6$. This technique will
be used in future experiments, namely with the James Webb Space Telescope, to further
constrain the neutral fraction of the IGM across cosmic time. The most up to date constraints on reionization including
Planck and high-redshift galaxy measurements can be found in Figure \ref{fig:reionization_constraints}.

\begin{figure}[th]
	\centering
	\includegraphics[width=0.7\textwidth]{tyler_reionization_history.pdf}
	\caption[Reionization Constraints]{Constraints on reionization. Modified from \cite{2019arXiv191103499W}}
	\label{fig:reionization_constraints}
\end{figure}


\begin{figure}[th]
	\centering
	\includegraphics[width=0.90\textwidth]{spin_flip_H.png}
	\caption[Spin-Flip Transition of Neutral Hydrogen]{A depiction of the spin-flip transition of neutral hydrogen. Initially,
																					 the spin of the proton and electron are parallel and oriented
																					 in the same direction. The transition occurs when the electron's spin spontaneously
																					 flips from the higher energy parallel alignment to the lower energy anti-parallel alignment,
																					 releasing a photon with a wavelength of 21\,cm.}
	\label{fig:spin_flip}
\end{figure}

In addition to CMB and high-redshift quasar measurements, other techniques are being
developed to understand the progression of reionization. In particular, spectral line
intensity mapping looks to be promising method of studying the EoR by constructing
three dimensional maps of the universe using well-known spectral lines
(21\,cm hyperfine transition \ref{fig:spin_flip}, \lya, CII, CO, etc.). The major advantage
of this method is that these spectral lines are narrow and well understood so observed redshift
maps directly translate to an observation of a specific cosmic time. This
method promises to place some of the tightest constraints on reionization by capturing
the behavior of the first luminous objects and the IGM on large scales.

\begin{figure}[th]
	\centering
	\includegraphics[width=0.95\textwidth]{results/intensity_mapping.png}
	\caption[Redshift and Resolution Coverage of Intensity Mapping Experiments]{A figure
					representing the resolution, field of view, and redshift range of upcoming intensity mapping
					experiments.}
	\label{fig:intensity_mapping}
\end{figure}

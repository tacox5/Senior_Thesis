\begin{equation}
    k_{\parallel} \leq \sin \left( \theta_0 \right) \frac{H_0 E \left( z \right) D_{M}\left( z \right)}{c \left(1 + z\right)} k_{\perp}
\end{equation}

Here is an explanation of the wedge

Cross correlation of cosmological 21cm observations with Ly$\alpha$ intensity
mapping surveys have the advantage that 21cm foregrounds do not correlation with
the foregrounds. Because the two do not correlate, no power from the foregrounds
is added to the cross-power spectrum. However, while the foregrounds don't contribute
to the total cross-power spectrum, they do contribute to the overall variance
of the measurement, and therefore the errors.

In the previous section, I calculated the cross-power spectrum for the case where
foreground are completely uncorrelated (don't contribute to the cross-power spectrum
amplitude) and where they did not contribute to the total variance. In this section,
I'll take a more realistic and honest approach to calculating the errors by dealing
with the 21cm foregrounds in two different ways. There are two primary methods
that I chose to deal with the foregrounds: by relying on the fact that the foregrounds
are spectrally smooth and thus confined to an area of the 2D power spectrum known
as the wedge and by assuming some imperfect removal method. Each method has advantages
and disadvantages.

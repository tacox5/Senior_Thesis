In order to more realistically estimate the observed \lya\ emission from galaxies,
attenuation of \lya\ by the IGM needs to be taken into account. I use the expression
below to calculate the \lya\ luminosity from galaxies actually observed,
\begin{equation}
L^{\textrm{gal}}_{\textrm{obs}} = L^{\textrm{gal}} e^{-\tau_{\textrm{Ly}\alpha}}.
\end{equation}
Here $L^{\textrm{gal}}$ is the luminosity of \lya\ emitting galaxies defined in the previous section and
$\tau_{\textrm{Ly}\alpha}$ is the optical depth of \lya\ emission at some redshift.

The general picture of the attenuation of \lya\ radiation from galaxies is as follows:
\lya\ radiation is emitted by some source and as that radiation travels from the virial
radius of that \lya\ emitter to the edge of the ionized bubble, it is redshifted
out of resonance and into the line damping wings, where it has a lower probability of
being absorbed. As reionization progresses and the ionized bubbles around halos grow,
the probability of \lya\ being attenuated decreases given the fact that it redshifts more
for reaching the neutral hydrogen. To simulate this behavior, I use a
model for the optical depth of \lya\ that is defined in \cite{1998ApJ...501...15M}. The
key equation associated with this model is given by,
\begin{align}
\tau_{\textrm{Ly}\alpha} &= \tau_{\textrm{s}} \bar{x}_{\textrm{HI}} \left( \frac{2.02 \times 10^{-8}}{\pi}\right) \left(\frac{1 + z_s}{1 + z_{\textrm{obs}}} \right)^{1.5} \nonumber \\
& \hspace{1em} \times \left[ I\left(\frac{1 + z_s}{1 + z_{\textrm{obs}}} \right) - I\left(\frac{1 + z_{\textrm{reion}}}{1 + z_{\textrm{obs}}}\right) \right]
\end{align}
where $z_{s}$ is the redshift of a \lya\ emitting source, $z_{\rm obs}$ is the redshift of the observation,
, and $\bar{x}_{\textrm{HI}}$, is the mean neutral fraction. This assumes that
the optical line depth at \lya\ line resonance, $\tau_s$, can be approximated as
\begin{equation}
\tau_{\textrm{s}} \approx  6.45 \times 10^5 \left( \frac{\Omega_b h}{0.03}\right) \left( \frac{\Omega_m}{0.3}\right)^{-0.5} \left(  \frac{1 + z_{\textrm{s}}}{10}\right)^{1.5}
\end{equation}
at high-redshifts for a source at some redshift, $z_{s}$, given present day, $\Omega_b$, and $\Omega_m$ (\cite{1965ApJ...142.1633G}, \cite{2001PhR...349..125B}). In the expression
above, $I$ is the helper function,
\begin{align}
I \left(x \right) &= \frac{x^{4.5}}{1 - x} + \frac{9}{7}x^{3.5} + \frac{9}{5}x^{2.5} + 3 x^{1.5} + 9 x^{0.5} \nonumber \\
                & \hspace{1em}- 4.5 \log \left( \frac{1 + x^{0.5}}{1 - x^{0.5}}\right).
\end{align}

To calculate the difference between $z_s$ and $z_{\rm obs}$, I estimated the mean ionized bubble
size at some $z_{\rm obs}$ by counting the number of voxels between each halo and the nearest
voxel which contained neutral gas and averaging that value over all halos. I then scaled this value
by the comoving voxel size to obtain an estimate on the mean ionized bubble size at $z_{\rm obs}$. This was then
converted to a $z_s$ value using the relationship $z_{s} = z_{obs} - H\left(z_{obs} \right) D / c$, where
$D$ is the typical ionized bubble size. As expected, this model results in greater attenuation
at high-redshifts and decreases as the ionized bubbles grow and the mean neutral fraction decreases.
The effect of the attentuation resulted in \lya\ transmission values of
$e^{-\tau_{\textrm{Ly}\alpha}} \approx 0.98,\, 0.82,\, 0.59$ at $z = 6, \, 7,\, 8$.

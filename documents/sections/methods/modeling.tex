In this section, I will discuss the effort to model the 21\,cm and \lya\
cosmological signals that will be used for cross-correlation later in this work.
A significant amount of research has gone into the modeling effort of both 21\,cm and
\lya\ fluctuations during reionization. While N-body/radiative
transfer codes accurately capture the physics of the evolution of
galaxies and the IGM, they are computationally
expensive and are difficult to extend to the large cosmological volumes that HERA and SPHEREx will
attempt to observe. Semi-numerical simulators on the other hand are much more
computationally efficient at modeling the large-scale evolution of the IGM and
strongly agree with more numerically motivated simulators at the large scales that
these intensity mapping experiments are concerned with.
Additionally, computational efficiency is not only more convenient, but necessary
for later parameter studies.
For this work, I make use of the semi-numerical
simulator, \fastsim\ (\cite{2011MNRAS.411..955M}),
to simulate 21\,cm emission and to generate the density and ionization fields and halo catalogue
necessary for the simulation of \lya\ emission in later subsections.

Throughout this is work, I assume a flat, $\Lambda$CDM cosmology using cosmological
parameters consistent with \cite{2016A&A...594A..13P} and astrophysical
parameters defined in the \fastsim\ fidicual model. The physical comoving size of the
simulation cubes was $200 \times 200 \times 200 \ {\rm Mpc}^3$ with a voxel resolution
of 256 voxels on a side.

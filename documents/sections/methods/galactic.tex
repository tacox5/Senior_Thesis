\label{ref:laes}

I start by describing the emission of EoR \lya\ stemming from \lya\ emitting
galaxies. As mentioned above, the emission of \lya\ photons in LAE's is a result
of two dominant processes: the recombination of ionized hydrogen within the virial
radius of the halos and collisional excitations of neutral hydrogen. \cite{2013ApJ...763..132S}
also modeled two other contributions to \lya\ emission within these halos but
found them to be subdominant, so for this work I just focus on these two main sources.
Both of these dominant sources of \lya\ emission are closely related to star formation and therefore
will be dependent star formation rate (SFR) of the LAEs, which is parameterized
by their halo mass.

I'll begin by discussing \lya\ emission stemming from recombinations of ionized hydrogen.
The expression for the \lya\ luminosity of a galaxy with mass, $M$, at redshift, $z$ is defined as,
\begin{equation}
  L_{\rm rec}^{\rm gal} \left( M, z\right) = E_{\rm Ly\alpha} \dot{N}_{\rm Ly\alpha} \left( M, z\right),
\end{equation}
where I assume the emission of \lya\ radiation at rest-frame frequency
with energy $E_{\rm Ly\alpha} = 13.6 \ {\rm eV}$. The relationship above gives the
luminosity of \lya\ emission from galaxies by calculating the rate of \lya\
photons being emitted and multiplying it by the energy of \lya\ photons. The number of
\lya\ photons being emitted per second, $\dot{N}_{\rm Ly\alpha}$, can be estimated
using the relationship below,

\begin{equation}
  \dot{N}_{\rm Ly\alpha} \left( M, z\right) = A_{\rm He} \dot{N}_{\rm ion} f_{\rm rec} f_{\rm Ly\alpha} \left[ 1 - f_{\rm esc} \left( M, z\right) \right].
\end{equation}

Here $A_{\rm He} = \left(4 - Y_{\rm He} \right) / \left( 4 - 3 Y_{\rm He} \right)$ (for helium mass fraction $Y_{\rm He} = 0.24$) and
the fraction of recombinations which result in the emission of a \lya\ photon, $f_{\rm rec} \approx 0.66$.
The fraction of \lya\ photons not absorbed
by dust is parameterized in \cite{2011ApJ...730....8H} by the expression,
\begin{equation}
f_{\rm Ly\alpha} = C_{\rm dust} \times 10^{-3} \left( 1 + z\right)^{\zeta}
\end{equation}
where $C_{\rm dust} = 3.34$ and $\zeta = 2.57$. Below, the relationship for the
escape fraction of \lya\ radiation is given as,
\begin{equation}
f_{\rm esc} \left( M, z\right) = \exp \left[- \alpha \left(z \right) M^{\beta \left(z \right)} \right],
\end{equation}
where $\alpha$ and $\beta$ are redshift dependent parameters defined in \cite{2010ApJ...710.1239R}.
The key relationship in this model is the star formation rate (SFR)
which is parameterized by the mass of the halo,
\begin{equation}
  {\rm SFR} = 2.8 \times 10^{-28} \left(\frac{M}{M_{\odot}} \right)^a \left( 1 + \frac{M}{c_1}\right)^b \left(1 + \frac{M}{c_2} \right)^d \ \left[ M_{\odot} \ {\rm yr}^{-1} \right].
\end{equation}
In this relationship, $a = -0.94$, $b = -1.7$, $c_1 = 10^9 \ M_{\odot}$, $c_2 = 7 \times 10^{10} \ M_{\odot}$, and $d = -1.7$
are fitted parameters defined in (\cite{2013ApJ...763..132S}). Finally, the total number of ionizing photons is defined as $\dot{N}_{\rm ion} = Q_{\rm ion} \times \textrm{SFR} \left(M, z \right)$ where $Q_{\rm ion} \approx 5.8 \times 10^{60} \ M_{\odot}^{-1}$
(\cite{2002A&A...382...28S}). The value for $Q_{\rm ion}$ was found by modeling stellar
lifetimes and estimating the number of ionizing photons per unit time.

The other dominant contributor of \lya\ emission in galaxies is excitation during
hydrogen ionization. The \lya\ luminosity in the interstellar medium due to these
excitations is defined as,
\begin{equation}
  L_{\rm exc}^{\rm gal} \left( M, z\right) = A_{\rm He} f_{\rm Ly\alpha} \left[ 1 - f_{\rm esc} \left( M, z\right) \right] E_{\rm exc} \dot{N}_{\rm ion},
\end{equation}
where all terms have been previously defined, with the exception of $E_{\rm exc} \approx 2.14 {\rm eV}$, which
was determined by estimating the average ionizing photon energy for thermal equilibrium, $E_{\nu} = 21.4$ eV, (\cite{2005MNRAS.362..799M})
and relating it to the energy emitted as \lya\ radiation due to excitations, $E_{\rm exc} / E_{\nu} \approx 0.1$ (\cite{1996ApJ...468..462G}).
With both dominant contributions to \lya\ emission modeled, I assign a \lya\ luminosity to each
halo dependent on its mass using the parameterization above. These luminosities can then be converted to a surface brightness by
dividing them by the simulation comoving voxel size and using the following expression,
\begin{equation}
  I_{\nu}^{\rm gal}\left(\textbf{x}, z \right) = y \left(z \right)d_A^2 \left(z \right) \frac{L^{\rm gal} \left(\textbf{x}, z \right)}{4 \pi d_L^2}.
\end{equation}
Here $d_A$ is the comoving angular diameter distance, $d_L$ is the luminosity distance, and
$y \left( z \right) = \lambda_{0} \left( 1 + z\right)^2 / H \left(z \right)$ (for the rest-frame
wavelength of \lya\ radiation, $\lambda_0 = 1216$ \AA). With these surface brightnesses calculated,
a cube of \lya\ emission from galaxies can then be constructed by adding the emission from
galaxies to their cooresponding voxel position, \textbf{x}. This produces
a cube of \lya\ emission from galaxies that is scaled by their masses and follows the
spatial distribution of the halos. A slice of these simulation cubes across the redshift
range of interest can be found in Figure \ref{fig:sims}.

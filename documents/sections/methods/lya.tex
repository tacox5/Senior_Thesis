Here I will describe the parameterization of \lya\ emission during reionization.
I adopt the techique developed in \cite{2013ApJ...763..132S} and first applied to \fastsim\ simulations
in \cite{2017ApJ...848...52H} for simulating \lya\ intensity mapping measurements.
While those works already provide in-depth descriptions of the relationships used to simulate
\lya\ emission, I will restate those methods in the following sections. Using this
procedure, \lya\ fluctuations can be modeled from two distinct sources. They are:

\begin{enumerate}
\item \textit{\lya\ Emitters (LAE)}: This is emission that originates from within the
              virial radius \lya\-emitting galaxies themselves. The dominate components
              that contribute to emission within these galaxies are hydrogen recombinations
              and collisional excitation.
\item \textit{Ionized Intergalactic Medium (IGM)}: This is emission that stems from the bubble
              of ionized gas that surround \lya\-emitting galaxies. In these ionized bubbles,
              recombinations of ionized hydrogen is the dominate contributor to \lya\ emission.
\end{enumerate}

In the following sections, I will describe the procedure for simulating \lya\
emission from these two sources during reionization using \fastsim\ halo catalogs
and kinetic gas temperature, non-linear density contrast, and ionization cubes.

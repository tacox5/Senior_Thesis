The Subaru Strategic Exploration of Exoplanets and Disks (SEEDS) collaboration is conducting high-contrast observations of exoplanets and disks using the 188 actu- ator AO system (AO188), classical Lyot coronagraph, and near infrared differential imaging science camera (HiCIAO) on the Subaru Telescope. This work supports the SEEDS project by conducting an analysis of the importance of field rotation for observations acquired in angular differential mode (ADI). ADI suppresses optical speckle features by fixing the telescope point spread function (PSF) with respect to the science camera detector, while allowing the field with any companions to rotate freely. In post processing, the sequence of images are registered and combined to produce a reference PSF that is subtracted from each image. The final images are de-rotated to realign the field with a fixed position angle. We find that increasing field rotation does influence sensitivity and that doubling the amount of field rotation in an observation can reduce the minimum angular separation at which companions can be detected by 0.1" and increase the sensitivity of an observation by factors of ~2–4 at separations of ~0.2"–0.4". These results emphasize the importance of conducting ADI observations during transit when the rotation rate of the target is maximized.
\newpage
